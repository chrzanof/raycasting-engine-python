%% Generated by Sphinx.
\def\sphinxdocclass{report}
\documentclass[letterpaper,10pt,english]{sphinxmanual}
\ifdefined\pdfpxdimen
   \let\sphinxpxdimen\pdfpxdimen\else\newdimen\sphinxpxdimen
\fi \sphinxpxdimen=.75bp\relax
\ifdefined\pdfimageresolution
    \pdfimageresolution= \numexpr \dimexpr1in\relax/\sphinxpxdimen\relax
\fi
%% let collapsible pdf bookmarks panel have high depth per default
\PassOptionsToPackage{bookmarksdepth=5}{hyperref}

\PassOptionsToPackage{booktabs}{sphinx}
\PassOptionsToPackage{colorrows}{sphinx}

\PassOptionsToPackage{warn}{textcomp}
\usepackage[utf8]{inputenc}
\ifdefined\DeclareUnicodeCharacter
% support both utf8 and utf8x syntaxes
  \ifdefined\DeclareUnicodeCharacterAsOptional
    \def\sphinxDUC#1{\DeclareUnicodeCharacter{"#1}}
  \else
    \let\sphinxDUC\DeclareUnicodeCharacter
  \fi
  \sphinxDUC{00A0}{\nobreakspace}
  \sphinxDUC{2500}{\sphinxunichar{2500}}
  \sphinxDUC{2502}{\sphinxunichar{2502}}
  \sphinxDUC{2514}{\sphinxunichar{2514}}
  \sphinxDUC{251C}{\sphinxunichar{251C}}
  \sphinxDUC{2572}{\textbackslash}
\fi
\usepackage{cmap}
\usepackage[T1]{fontenc}
\usepackage{amsmath,amssymb,amstext}
\usepackage{babel}



\usepackage{tgtermes}
\usepackage{tgheros}
\renewcommand{\ttdefault}{txtt}



\usepackage[Bjarne]{fncychap}
\usepackage{sphinx}

\fvset{fontsize=auto}
\usepackage{geometry}


% Include hyperref last.
\usepackage{hyperref}
% Fix anchor placement for figures with captions.
\usepackage{hypcap}% it must be loaded after hyperref.
% Set up styles of URL: it should be placed after hyperref.
\urlstyle{same}

\addto\captionsenglish{\renewcommand{\contentsname}{Contents:}}

\usepackage{sphinxmessages}
\setcounter{tocdepth}{1}



\title{Raycasting Engine}
\date{Jun 13, 2023}
\release{1.0.0}
\author{Filip Chrzanowski}
\newcommand{\sphinxlogo}{\vbox{}}
\renewcommand{\releasename}{Release}
\makeindex
\begin{document}

\ifdefined\shorthandoff
  \ifnum\catcode`\=\string=\active\shorthandoff{=}\fi
  \ifnum\catcode`\"=\active\shorthandoff{"}\fi
\fi

\pagestyle{empty}
\sphinxmaketitle
\pagestyle{plain}
\sphinxtableofcontents
\pagestyle{normal}
\phantomsection\label{\detokenize{index::doc}}


\sphinxstepscope


\chapter{raycasting\sphinxhyphen{}engine\sphinxhyphen{}python}
\label{\detokenize{modules:raycasting-engine-python}}\label{\detokenize{modules::doc}}
\sphinxstepscope


\section{GameEngine module}
\label{\detokenize{GameEngine:module-GameEngine}}\label{\detokenize{GameEngine:gameengine-module}}\label{\detokenize{GameEngine::doc}}\index{module@\spxentry{module}!GameEngine@\spxentry{GameEngine}}\index{GameEngine@\spxentry{GameEngine}!module@\spxentry{module}}\index{GameEngine (class in GameEngine)@\spxentry{GameEngine}\spxextra{class in GameEngine}}

\begin{fulllineitems}
\phantomsection\label{\detokenize{GameEngine:GameEngine.GameEngine}}
\pysigstartsignatures
\pysiglinewithargsret{\sphinxbfcode{\sphinxupquote{class\DUrole{w,w}{  }}}\sphinxcode{\sphinxupquote{GameEngine.}}\sphinxbfcode{\sphinxupquote{GameEngine}}}{\sphinxparam{\DUrole{n,n}{width}}, \sphinxparam{\DUrole{n,n}{height}}, \sphinxparam{\DUrole{n,n}{level\_map}}, \sphinxparam{\DUrole{n,n}{player}}, \sphinxparam{\DUrole{n,n}{window}}, \sphinxparam{\DUrole{n,n}{textures}}, \sphinxparam{\DUrole{n,n}{sprites}}}{}
\pysigstopsignatures
\sphinxAtStartPar
Bases: \sphinxcode{\sphinxupquote{object}}
\index{render() (GameEngine.GameEngine method)@\spxentry{render()}\spxextra{GameEngine.GameEngine method}}

\begin{fulllineitems}
\phantomsection\label{\detokenize{GameEngine:GameEngine.GameEngine.render}}
\pysigstartsignatures
\pysiglinewithargsret{\sphinxbfcode{\sphinxupquote{render}}}{\sphinxparam{\DUrole{n,n}{canvas}}}{}
\pysigstopsignatures
\sphinxAtStartPar
rendering all entities
:param canvas: canvas containing all objects on the screen
:return canvas:

\end{fulllineitems}

\index{update() (GameEngine.GameEngine method)@\spxentry{update()}\spxextra{GameEngine.GameEngine method}}

\begin{fulllineitems}
\phantomsection\label{\detokenize{GameEngine:GameEngine.GameEngine.update}}
\pysigstartsignatures
\pysiglinewithargsret{\sphinxbfcode{\sphinxupquote{update}}}{}{}
\pysigstopsignatures
\sphinxAtStartPar
executing commands, updating all entities in a game and checking collisions

\end{fulllineitems}


\end{fulllineitems}


\sphinxstepscope


\section{InputHandler module}
\label{\detokenize{InputHandler:module-InputHandler}}\label{\detokenize{InputHandler:inputhandler-module}}\label{\detokenize{InputHandler::doc}}\index{module@\spxentry{module}!InputHandler@\spxentry{InputHandler}}\index{InputHandler@\spxentry{InputHandler}!module@\spxentry{module}}\index{InputHandler (class in InputHandler)@\spxentry{InputHandler}\spxextra{class in InputHandler}}

\begin{fulllineitems}
\phantomsection\label{\detokenize{InputHandler:InputHandler.InputHandler}}
\pysigstartsignatures
\pysiglinewithargsret{\sphinxbfcode{\sphinxupquote{class\DUrole{w,w}{  }}}\sphinxcode{\sphinxupquote{InputHandler.}}\sphinxbfcode{\sphinxupquote{InputHandler}}}{\sphinxparam{\DUrole{n,n}{window}}, \sphinxparam{\DUrole{n,n}{actor}}}{}
\pysigstopsignatures
\sphinxAtStartPar
Bases: \sphinxcode{\sphinxupquote{object}}

\sphinxAtStartPar
class for handling input from keyboard
\index{handle\_input() (InputHandler.InputHandler method)@\spxentry{handle\_input()}\spxextra{InputHandler.InputHandler method}}

\begin{fulllineitems}
\phantomsection\label{\detokenize{InputHandler:InputHandler.InputHandler.handle_input}}
\pysigstartsignatures
\pysiglinewithargsret{\sphinxbfcode{\sphinxupquote{handle\_input}}}{}{}
\pysigstopsignatures\begin{description}
\sphinxlineitem{this method checks which key is present in input\_buffer and returns command\_buffer array that contains command}
\sphinxAtStartPar
objects

\end{description}
\begin{quote}\begin{description}
\sphinxlineitem{Returns}
\sphinxAtStartPar
command buffer

\end{description}\end{quote}

\end{fulllineitems}

\index{keydown() (InputHandler.InputHandler method)@\spxentry{keydown()}\spxextra{InputHandler.InputHandler method}}

\begin{fulllineitems}
\phantomsection\label{\detokenize{InputHandler:InputHandler.InputHandler.keydown}}
\pysigstartsignatures
\pysiglinewithargsret{\sphinxbfcode{\sphinxupquote{keydown}}}{\sphinxparam{\DUrole{n,n}{e}}}{}
\pysigstopsignatures
\sphinxAtStartPar
detects key being pressed and adds it to input\_buffer
:param e: event

\end{fulllineitems}

\index{keyup() (InputHandler.InputHandler method)@\spxentry{keyup()}\spxextra{InputHandler.InputHandler method}}

\begin{fulllineitems}
\phantomsection\label{\detokenize{InputHandler:InputHandler.InputHandler.keyup}}
\pysigstartsignatures
\pysiglinewithargsret{\sphinxbfcode{\sphinxupquote{keyup}}}{\sphinxparam{\DUrole{n,n}{e}}}{}
\pysigstopsignatures
\sphinxAtStartPar
detects key being released and removes it from input\_buffer
:param e: event

\end{fulllineitems}


\end{fulllineitems}


\sphinxstepscope


\section{RaycastingEngine module}
\label{\detokenize{RaycastingEngine:module-RaycastingEngine}}\label{\detokenize{RaycastingEngine:raycastingengine-module}}\label{\detokenize{RaycastingEngine::doc}}\index{module@\spxentry{module}!RaycastingEngine@\spxentry{RaycastingEngine}}\index{RaycastingEngine@\spxentry{RaycastingEngine}!module@\spxentry{module}}\index{RaycastingEngine (class in RaycastingEngine)@\spxentry{RaycastingEngine}\spxextra{class in RaycastingEngine}}

\begin{fulllineitems}
\phantomsection\label{\detokenize{RaycastingEngine:RaycastingEngine.RaycastingEngine}}
\pysigstartsignatures
\pysiglinewithargsret{\sphinxbfcode{\sphinxupquote{class\DUrole{w,w}{  }}}\sphinxcode{\sphinxupquote{RaycastingEngine.}}\sphinxbfcode{\sphinxupquote{RaycastingEngine}}}{\sphinxparam{\DUrole{n,n}{width}}, \sphinxparam{\DUrole{n,n}{height}}, \sphinxparam{\DUrole{n,n}{level}}, \sphinxparam{\DUrole{n,n}{player}}, \sphinxparam{\DUrole{n,n}{textures}}, \sphinxparam{\DUrole{n,n}{sprites}}}{}
\pysigstopsignatures
\sphinxAtStartPar
Bases: \sphinxcode{\sphinxupquote{object}}

\sphinxAtStartPar
class responsible for 3D (2.5D actually) view
\index{calculate\_sprite\_screen\_parameters() (RaycastingEngine.RaycastingEngine method)@\spxentry{calculate\_sprite\_screen\_parameters()}\spxextra{RaycastingEngine.RaycastingEngine method}}

\begin{fulllineitems}
\phantomsection\label{\detokenize{RaycastingEngine:RaycastingEngine.RaycastingEngine.calculate_sprite_screen_parameters}}
\pysigstartsignatures
\pysiglinewithargsret{\sphinxbfcode{\sphinxupquote{calculate\_sprite\_screen\_parameters}}}{\sphinxparam{\DUrole{n,n}{sprite}}}{}
\pysigstopsignatures
\sphinxAtStartPar
calculating all parameters necessary to render sprite on screen
:param sprite:
:return: screen x coordinate, width, height, brightness modifier, isVisible, distance to player

\end{fulllineitems}

\index{cast\_rays() (RaycastingEngine.RaycastingEngine method)@\spxentry{cast\_rays()}\spxextra{RaycastingEngine.RaycastingEngine method}}

\begin{fulllineitems}
\phantomsection\label{\detokenize{RaycastingEngine:RaycastingEngine.RaycastingEngine.cast_rays}}
\pysigstartsignatures
\pysiglinewithargsret{\sphinxbfcode{\sphinxupquote{cast\_rays}}}{}{}
\pysigstopsignatures
\sphinxAtStartPar
method performing a ray casting algorithm
:return: array of textured stripes

\end{fulllineitems}

\index{check\_ray\_collision() (RaycastingEngine.RaycastingEngine method)@\spxentry{check\_ray\_collision()}\spxextra{RaycastingEngine.RaycastingEngine method}}

\begin{fulllineitems}
\phantomsection\label{\detokenize{RaycastingEngine:RaycastingEngine.RaycastingEngine.check_ray_collision}}
\pysigstartsignatures
\pysiglinewithargsret{\sphinxbfcode{\sphinxupquote{check\_ray\_collision}}}{\sphinxparam{\DUrole{n,n}{ray\_angle}}, \sphinxparam{\DUrole{n,n}{player\_x}}, \sphinxparam{\DUrole{n,n}{player\_y}}, \sphinxparam{\DUrole{n,n}{level}}}{}
\pysigstopsignatures
\sphinxAtStartPar
casting a single ray
:param ray\_angle: angle of the ray relative to the world
:param player\_x: player x coordinate
:param player\_y: player y coordinate
:param level: level matrix \sphinxhyphen{} zeros represents empty space, any number higher than 0 is a texture id
:return: ray length, wall hit\sphinxhyphen{}point x coordinate, wall hit\sphinxhyphen{}point y coordinate, index of texture

\end{fulllineitems}

\index{render() (RaycastingEngine.RaycastingEngine method)@\spxentry{render()}\spxextra{RaycastingEngine.RaycastingEngine method}}

\begin{fulllineitems}
\phantomsection\label{\detokenize{RaycastingEngine:RaycastingEngine.RaycastingEngine.render}}
\pysigstartsignatures
\pysiglinewithargsret{\sphinxbfcode{\sphinxupquote{render}}}{\sphinxparam{\DUrole{n,n}{canvas}}}{}
\pysigstopsignatures
\sphinxAtStartPar
rendering objects in the furthest to the nearest order
:param canvas: canvas
:return: canvas: updated canvas

\end{fulllineitems}


\end{fulllineitems}


\sphinxstepscope


\section{SpriteRender module}
\label{\detokenize{SpriteRender:module-SpriteRender}}\label{\detokenize{SpriteRender:spriterender-module}}\label{\detokenize{SpriteRender::doc}}\index{module@\spxentry{module}!SpriteRender@\spxentry{SpriteRender}}\index{SpriteRender@\spxentry{SpriteRender}!module@\spxentry{module}}\index{SpriteRender (class in SpriteRender)@\spxentry{SpriteRender}\spxextra{class in SpriteRender}}

\begin{fulllineitems}
\phantomsection\label{\detokenize{SpriteRender:SpriteRender.SpriteRender}}
\pysigstartsignatures
\pysiglinewithargsret{\sphinxbfcode{\sphinxupquote{class\DUrole{w,w}{  }}}\sphinxcode{\sphinxupquote{SpriteRender.}}\sphinxbfcode{\sphinxupquote{SpriteRender}}}{\sphinxparam{\DUrole{n,n}{params}}}{}
\pysigstopsignatures
\sphinxAtStartPar
Bases: \sphinxcode{\sphinxupquote{object}}

\sphinxAtStartPar
class encapsulating all params to render sprite
\index{render() (SpriteRender.SpriteRender method)@\spxentry{render()}\spxextra{SpriteRender.SpriteRender method}}

\begin{fulllineitems}
\phantomsection\label{\detokenize{SpriteRender:SpriteRender.SpriteRender.render}}
\pysigstartsignatures
\pysiglinewithargsret{\sphinxbfcode{\sphinxupquote{render}}}{\sphinxparam{\DUrole{n,n}{canvas}}}{}
\pysigstopsignatures
\end{fulllineitems}


\end{fulllineitems}


\sphinxstepscope


\section{commands package}
\label{\detokenize{commands:commands-package}}\label{\detokenize{commands::doc}}

\subsection{Submodules}
\label{\detokenize{commands:submodules}}

\subsection{commands.ChangeWeaponCommand module}
\label{\detokenize{commands:module-commands.ChangeWeaponCommand}}\label{\detokenize{commands:commands-changeweaponcommand-module}}\index{module@\spxentry{module}!commands.ChangeWeaponCommand@\spxentry{commands.ChangeWeaponCommand}}\index{commands.ChangeWeaponCommand@\spxentry{commands.ChangeWeaponCommand}!module@\spxentry{module}}\index{ChangeWeaponCommand (class in commands.ChangeWeaponCommand)@\spxentry{ChangeWeaponCommand}\spxextra{class in commands.ChangeWeaponCommand}}

\begin{fulllineitems}
\phantomsection\label{\detokenize{commands:commands.ChangeWeaponCommand.ChangeWeaponCommand}}
\pysigstartsignatures
\pysiglinewithargsret{\sphinxbfcode{\sphinxupquote{class\DUrole{w,w}{  }}}\sphinxcode{\sphinxupquote{commands.ChangeWeaponCommand.}}\sphinxbfcode{\sphinxupquote{ChangeWeaponCommand}}}{\sphinxparam{\DUrole{n,n}{player}\DUrole{p,p}{:}\DUrole{w,w}{  }\DUrole{n,n}{{\hyperref[\detokenize{objects:objects.Player.Player}]{\sphinxcrossref{Player}}}}}, \sphinxparam{\DUrole{n,n}{index}\DUrole{p,p}{:}\DUrole{w,w}{  }\DUrole{n,n}{int}}}{}
\pysigstopsignatures
\sphinxAtStartPar
Bases: {\hyperref[\detokenize{commands:commands.Command.Command}]{\sphinxcrossref{\sphinxcode{\sphinxupquote{Command}}}}}
\index{execute() (commands.ChangeWeaponCommand.ChangeWeaponCommand method)@\spxentry{execute()}\spxextra{commands.ChangeWeaponCommand.ChangeWeaponCommand method}}

\begin{fulllineitems}
\phantomsection\label{\detokenize{commands:commands.ChangeWeaponCommand.ChangeWeaponCommand.execute}}
\pysigstartsignatures
\pysiglinewithargsret{\sphinxbfcode{\sphinxupquote{execute}}}{}{}
\pysigstopsignatures
\sphinxAtStartPar
changes weapon to a given index
:return:

\end{fulllineitems}

\index{undo() (commands.ChangeWeaponCommand.ChangeWeaponCommand method)@\spxentry{undo()}\spxextra{commands.ChangeWeaponCommand.ChangeWeaponCommand method}}

\begin{fulllineitems}
\phantomsection\label{\detokenize{commands:commands.ChangeWeaponCommand.ChangeWeaponCommand.undo}}
\pysigstartsignatures
\pysiglinewithargsret{\sphinxbfcode{\sphinxupquote{undo}}}{}{}
\pysigstopsignatures
\sphinxAtStartPar
undoes the command
:return:

\end{fulllineitems}


\end{fulllineitems}



\subsection{commands.Command module}
\label{\detokenize{commands:module-commands.Command}}\label{\detokenize{commands:commands-command-module}}\index{module@\spxentry{module}!commands.Command@\spxentry{commands.Command}}\index{commands.Command@\spxentry{commands.Command}!module@\spxentry{module}}\index{Command (class in commands.Command)@\spxentry{Command}\spxextra{class in commands.Command}}

\begin{fulllineitems}
\phantomsection\label{\detokenize{commands:commands.Command.Command}}
\pysigstartsignatures
\pysigline{\sphinxbfcode{\sphinxupquote{class\DUrole{w,w}{  }}}\sphinxcode{\sphinxupquote{commands.Command.}}\sphinxbfcode{\sphinxupquote{Command}}}
\pysigstopsignatures
\sphinxAtStartPar
Bases: \sphinxcode{\sphinxupquote{ABC}}

\sphinxAtStartPar
abstract class for all commands.
Implemented according to Command design pattern.
\index{execute() (commands.Command.Command method)@\spxentry{execute()}\spxextra{commands.Command.Command method}}

\begin{fulllineitems}
\phantomsection\label{\detokenize{commands:commands.Command.Command.execute}}
\pysigstartsignatures
\pysiglinewithargsret{\sphinxbfcode{\sphinxupquote{abstract\DUrole{w,w}{  }}}\sphinxbfcode{\sphinxupquote{execute}}}{}{}
\pysigstopsignatures
\sphinxAtStartPar
executes the command
:return:

\end{fulllineitems}

\index{undo() (commands.Command.Command method)@\spxentry{undo()}\spxextra{commands.Command.Command method}}

\begin{fulllineitems}
\phantomsection\label{\detokenize{commands:commands.Command.Command.undo}}
\pysigstartsignatures
\pysiglinewithargsret{\sphinxbfcode{\sphinxupquote{abstract\DUrole{w,w}{  }}}\sphinxbfcode{\sphinxupquote{undo}}}{}{}
\pysigstopsignatures
\sphinxAtStartPar
undoes the command
:return:

\end{fulllineitems}


\end{fulllineitems}



\subsection{commands.MoveActorCommand module}
\label{\detokenize{commands:module-commands.MoveActorCommand}}\label{\detokenize{commands:commands-moveactorcommand-module}}\index{module@\spxentry{module}!commands.MoveActorCommand@\spxentry{commands.MoveActorCommand}}\index{commands.MoveActorCommand@\spxentry{commands.MoveActorCommand}!module@\spxentry{module}}\index{MoveActorCommand (class in commands.MoveActorCommand)@\spxentry{MoveActorCommand}\spxextra{class in commands.MoveActorCommand}}

\begin{fulllineitems}
\phantomsection\label{\detokenize{commands:commands.MoveActorCommand.MoveActorCommand}}
\pysigstartsignatures
\pysiglinewithargsret{\sphinxbfcode{\sphinxupquote{class\DUrole{w,w}{  }}}\sphinxcode{\sphinxupquote{commands.MoveActorCommand.}}\sphinxbfcode{\sphinxupquote{MoveActorCommand}}}{\sphinxparam{\DUrole{n,n}{actor}\DUrole{p,p}{:}\DUrole{w,w}{  }\DUrole{n,n}{{\hyperref[\detokenize{objects:objects.Actor.Actor}]{\sphinxcrossref{Actor}}}}}, \sphinxparam{\DUrole{n,n}{dx}\DUrole{p,p}{:}\DUrole{w,w}{  }\DUrole{n,n}{float}}, \sphinxparam{\DUrole{n,n}{dy}\DUrole{p,p}{:}\DUrole{w,w}{  }\DUrole{n,n}{float}}}{}
\pysigstopsignatures
\sphinxAtStartPar
Bases: {\hyperref[\detokenize{commands:commands.Command.Command}]{\sphinxcrossref{\sphinxcode{\sphinxupquote{Command}}}}}
\index{execute() (commands.MoveActorCommand.MoveActorCommand method)@\spxentry{execute()}\spxextra{commands.MoveActorCommand.MoveActorCommand method}}

\begin{fulllineitems}
\phantomsection\label{\detokenize{commands:commands.MoveActorCommand.MoveActorCommand.execute}}
\pysigstartsignatures
\pysiglinewithargsret{\sphinxbfcode{\sphinxupquote{execute}}}{}{}
\pysigstopsignatures
\sphinxAtStartPar
moves actor to new position
:return:

\end{fulllineitems}

\index{undo() (commands.MoveActorCommand.MoveActorCommand method)@\spxentry{undo()}\spxextra{commands.MoveActorCommand.MoveActorCommand method}}

\begin{fulllineitems}
\phantomsection\label{\detokenize{commands:commands.MoveActorCommand.MoveActorCommand.undo}}
\pysigstartsignatures
\pysiglinewithargsret{\sphinxbfcode{\sphinxupquote{undo}}}{}{}
\pysigstopsignatures
\sphinxAtStartPar
undoes the command
:return:

\end{fulllineitems}


\end{fulllineitems}



\subsection{commands.RotateActorCommand module}
\label{\detokenize{commands:module-commands.RotateActorCommand}}\label{\detokenize{commands:commands-rotateactorcommand-module}}\index{module@\spxentry{module}!commands.RotateActorCommand@\spxentry{commands.RotateActorCommand}}\index{commands.RotateActorCommand@\spxentry{commands.RotateActorCommand}!module@\spxentry{module}}\index{RotateActorCommand (class in commands.RotateActorCommand)@\spxentry{RotateActorCommand}\spxextra{class in commands.RotateActorCommand}}

\begin{fulllineitems}
\phantomsection\label{\detokenize{commands:commands.RotateActorCommand.RotateActorCommand}}
\pysigstartsignatures
\pysiglinewithargsret{\sphinxbfcode{\sphinxupquote{class\DUrole{w,w}{  }}}\sphinxcode{\sphinxupquote{commands.RotateActorCommand.}}\sphinxbfcode{\sphinxupquote{RotateActorCommand}}}{\sphinxparam{\DUrole{n,n}{actor}\DUrole{p,p}{:}\DUrole{w,w}{  }\DUrole{n,n}{{\hyperref[\detokenize{objects:objects.Actor.Actor}]{\sphinxcrossref{Actor}}}}}, \sphinxparam{\DUrole{n,n}{d\_angle}\DUrole{p,p}{:}\DUrole{w,w}{  }\DUrole{n,n}{float}}}{}
\pysigstopsignatures
\sphinxAtStartPar
Bases: {\hyperref[\detokenize{commands:commands.Command.Command}]{\sphinxcrossref{\sphinxcode{\sphinxupquote{Command}}}}}
\index{execute() (commands.RotateActorCommand.RotateActorCommand method)@\spxentry{execute()}\spxextra{commands.RotateActorCommand.RotateActorCommand method}}

\begin{fulllineitems}
\phantomsection\label{\detokenize{commands:commands.RotateActorCommand.RotateActorCommand.execute}}
\pysigstartsignatures
\pysiglinewithargsret{\sphinxbfcode{\sphinxupquote{execute}}}{}{}
\pysigstopsignatures
\sphinxAtStartPar
rotates actor to new angle
:return:

\end{fulllineitems}

\index{undo() (commands.RotateActorCommand.RotateActorCommand method)@\spxentry{undo()}\spxextra{commands.RotateActorCommand.RotateActorCommand method}}

\begin{fulllineitems}
\phantomsection\label{\detokenize{commands:commands.RotateActorCommand.RotateActorCommand.undo}}
\pysigstartsignatures
\pysiglinewithargsret{\sphinxbfcode{\sphinxupquote{undo}}}{}{}
\pysigstopsignatures
\sphinxAtStartPar
undoes the command
:return:

\end{fulllineitems}


\end{fulllineitems}



\subsection{commands.ZoomCommand module}
\label{\detokenize{commands:module-commands.ZoomCommand}}\label{\detokenize{commands:commands-zoomcommand-module}}\index{module@\spxentry{module}!commands.ZoomCommand@\spxentry{commands.ZoomCommand}}\index{commands.ZoomCommand@\spxentry{commands.ZoomCommand}!module@\spxentry{module}}\index{ZoomCommand (class in commands.ZoomCommand)@\spxentry{ZoomCommand}\spxextra{class in commands.ZoomCommand}}

\begin{fulllineitems}
\phantomsection\label{\detokenize{commands:commands.ZoomCommand.ZoomCommand}}
\pysigstartsignatures
\pysiglinewithargsret{\sphinxbfcode{\sphinxupquote{class\DUrole{w,w}{  }}}\sphinxcode{\sphinxupquote{commands.ZoomCommand.}}\sphinxbfcode{\sphinxupquote{ZoomCommand}}}{\sphinxparam{\DUrole{n,n}{actor}\DUrole{p,p}{:}\DUrole{w,w}{  }\DUrole{n,n}{{\hyperref[\detokenize{objects:objects.Actor.Actor}]{\sphinxcrossref{Actor}}}}}, \sphinxparam{\DUrole{n,n}{da}\DUrole{p,p}{:}\DUrole{w,w}{  }\DUrole{n,n}{float}}}{}
\pysigstopsignatures
\sphinxAtStartPar
Bases: {\hyperref[\detokenize{commands:commands.Command.Command}]{\sphinxcrossref{\sphinxcode{\sphinxupquote{Command}}}}}
\index{execute() (commands.ZoomCommand.ZoomCommand method)@\spxentry{execute()}\spxextra{commands.ZoomCommand.ZoomCommand method}}

\begin{fulllineitems}
\phantomsection\label{\detokenize{commands:commands.ZoomCommand.ZoomCommand.execute}}
\pysigstartsignatures
\pysiglinewithargsret{\sphinxbfcode{\sphinxupquote{execute}}}{}{}
\pysigstopsignatures
\sphinxAtStartPar
zooms the view by modifying actor’s vertical and horizontal field of view
:return:

\end{fulllineitems}

\index{undo() (commands.ZoomCommand.ZoomCommand method)@\spxentry{undo()}\spxextra{commands.ZoomCommand.ZoomCommand method}}

\begin{fulllineitems}
\phantomsection\label{\detokenize{commands:commands.ZoomCommand.ZoomCommand.undo}}
\pysigstartsignatures
\pysiglinewithargsret{\sphinxbfcode{\sphinxupquote{undo}}}{}{}
\pysigstopsignatures
\sphinxAtStartPar
undoes the command
:return:

\end{fulllineitems}


\end{fulllineitems}



\subsection{Module contents}
\label{\detokenize{commands:module-commands}}\label{\detokenize{commands:module-contents}}\index{module@\spxentry{module}!commands@\spxentry{commands}}\index{commands@\spxentry{commands}!module@\spxentry{module}}
\sphinxstepscope


\section{main module}
\label{\detokenize{main:main-module}}\label{\detokenize{main::doc}}
\sphinxstepscope


\section{objects package}
\label{\detokenize{objects:objects-package}}\label{\detokenize{objects::doc}}

\subsection{Submodules}
\label{\detokenize{objects:submodules}}

\subsection{objects.Actor module}
\label{\detokenize{objects:module-objects.Actor}}\label{\detokenize{objects:objects-actor-module}}\index{module@\spxentry{module}!objects.Actor@\spxentry{objects.Actor}}\index{objects.Actor@\spxentry{objects.Actor}!module@\spxentry{module}}\index{Actor (class in objects.Actor)@\spxentry{Actor}\spxextra{class in objects.Actor}}

\begin{fulllineitems}
\phantomsection\label{\detokenize{objects:objects.Actor.Actor}}
\pysigstartsignatures
\pysiglinewithargsret{\sphinxbfcode{\sphinxupquote{class\DUrole{w,w}{  }}}\sphinxcode{\sphinxupquote{objects.Actor.}}\sphinxbfcode{\sphinxupquote{Actor}}}{\sphinxparam{\DUrole{n,n}{x}}, \sphinxparam{\DUrole{n,n}{y}}, \sphinxparam{\DUrole{n,n}{angle}}, \sphinxparam{\DUrole{n,n}{speed}}, \sphinxparam{\DUrole{n,n}{rotation\_speed}}, \sphinxparam{\DUrole{n,n}{fov}}, \sphinxparam{\DUrole{n,n}{vertical\_angle}}, \sphinxparam{\DUrole{n,n}{vision\_distance}}, \sphinxparam{\DUrole{n,n}{radius}}}{}
\pysigstopsignatures
\sphinxAtStartPar
Bases: {\hyperref[\detokenize{objects:objects.GameObject.GameObject}]{\sphinxcrossref{\sphinxcode{\sphinxupquote{GameObject}}}}}
\index{move\_to() (objects.Actor.Actor method)@\spxentry{move\_to()}\spxextra{objects.Actor.Actor method}}

\begin{fulllineitems}
\phantomsection\label{\detokenize{objects:objects.Actor.Actor.move_to}}
\pysigstartsignatures
\pysiglinewithargsret{\sphinxbfcode{\sphinxupquote{move\_to}}}{\sphinxparam{\DUrole{n,n}{x}}, \sphinxparam{\DUrole{n,n}{y}}}{}
\pysigstopsignatures
\end{fulllineitems}

\index{render() (objects.Actor.Actor method)@\spxentry{render()}\spxextra{objects.Actor.Actor method}}

\begin{fulllineitems}
\phantomsection\label{\detokenize{objects:objects.Actor.Actor.render}}
\pysigstartsignatures
\pysiglinewithargsret{\sphinxbfcode{\sphinxupquote{render}}}{\sphinxparam{\DUrole{n,n}{canvas}}}{}
\pysigstopsignatures
\end{fulllineitems}

\index{rotate\_to() (objects.Actor.Actor method)@\spxentry{rotate\_to()}\spxextra{objects.Actor.Actor method}}

\begin{fulllineitems}
\phantomsection\label{\detokenize{objects:objects.Actor.Actor.rotate_to}}
\pysigstartsignatures
\pysiglinewithargsret{\sphinxbfcode{\sphinxupquote{rotate\_to}}}{\sphinxparam{\DUrole{n,n}{angle}}}{}
\pysigstopsignatures
\sphinxAtStartPar
updates angle of an actor in {[}0,2PI{]} range

\end{fulllineitems}

\index{update() (objects.Actor.Actor method)@\spxentry{update()}\spxextra{objects.Actor.Actor method}}

\begin{fulllineitems}
\phantomsection\label{\detokenize{objects:objects.Actor.Actor.update}}
\pysigstartsignatures
\pysiglinewithargsret{\sphinxbfcode{\sphinxupquote{update}}}{}{}
\pysigstopsignatures
\end{fulllineitems}


\end{fulllineitems}



\subsection{objects.DirectionalSprite module}
\label{\detokenize{objects:module-objects.DirectionalSprite}}\label{\detokenize{objects:objects-directionalsprite-module}}\index{module@\spxentry{module}!objects.DirectionalSprite@\spxentry{objects.DirectionalSprite}}\index{objects.DirectionalSprite@\spxentry{objects.DirectionalSprite}!module@\spxentry{module}}\index{DirectionalSprite (class in objects.DirectionalSprite)@\spxentry{DirectionalSprite}\spxextra{class in objects.DirectionalSprite}}

\begin{fulllineitems}
\phantomsection\label{\detokenize{objects:objects.DirectionalSprite.DirectionalSprite}}
\pysigstartsignatures
\pysiglinewithargsret{\sphinxbfcode{\sphinxupquote{class\DUrole{w,w}{  }}}\sphinxcode{\sphinxupquote{objects.DirectionalSprite.}}\sphinxbfcode{\sphinxupquote{DirectionalSprite}}}{\sphinxparam{\DUrole{n,n}{x}}, \sphinxparam{\DUrole{n,n}{y}}, \sphinxparam{\DUrole{n,n}{radius}}, \sphinxparam{\DUrole{n,n}{render\_radius}}, \sphinxparam{\DUrole{n,n}{angle}}, \sphinxparam{\DUrole{n,n}{folder\_path}}}{}
\pysigstopsignatures
\sphinxAtStartPar
Bases: {\hyperref[\detokenize{objects:objects.GameObject.GameObject}]{\sphinxcrossref{\sphinxcode{\sphinxupquote{GameObject}}}}}
\index{load\_images() (objects.DirectionalSprite.DirectionalSprite method)@\spxentry{load\_images()}\spxextra{objects.DirectionalSprite.DirectionalSprite method}}

\begin{fulllineitems}
\phantomsection\label{\detokenize{objects:objects.DirectionalSprite.DirectionalSprite.load_images}}
\pysigstartsignatures
\pysiglinewithargsret{\sphinxbfcode{\sphinxupquote{load\_images}}}{\sphinxparam{\DUrole{n,n}{folder\_path}}}{}
\pysigstopsignatures
\sphinxAtStartPar
loading sprite images
:param folder\_path: path to the folder containing images of sprite in 8 directions
:return:

\end{fulllineitems}

\index{render() (objects.DirectionalSprite.DirectionalSprite method)@\spxentry{render()}\spxextra{objects.DirectionalSprite.DirectionalSprite method}}

\begin{fulllineitems}
\phantomsection\label{\detokenize{objects:objects.DirectionalSprite.DirectionalSprite.render}}
\pysigstartsignatures
\pysiglinewithargsret{\sphinxbfcode{\sphinxupquote{render}}}{\sphinxparam{\DUrole{n,n}{canvas}}, \sphinxparam{\DUrole{n,n}{x}}, \sphinxparam{\DUrole{n,n}{y}}, \sphinxparam{\DUrole{n,n}{width}}, \sphinxparam{\DUrole{n,n}{height}}, \sphinxparam{\DUrole{n,n}{brightness}}}{}
\pysigstopsignatures
\sphinxAtStartPar
rendering sprite in current direction

\end{fulllineitems}

\index{update() (objects.DirectionalSprite.DirectionalSprite method)@\spxentry{update()}\spxextra{objects.DirectionalSprite.DirectionalSprite method}}

\begin{fulllineitems}
\phantomsection\label{\detokenize{objects:objects.DirectionalSprite.DirectionalSprite.update}}
\pysigstartsignatures
\pysiglinewithargsret{\sphinxbfcode{\sphinxupquote{update}}}{\sphinxparam{\DUrole{n,n}{player\_x}}, \sphinxparam{\DUrole{n,n}{player\_y}}, \sphinxparam{\DUrole{n,n}{player\_angle}}}{}
\pysigstopsignatures
\sphinxAtStartPar
updating sprite current direction based on angle between player facing direction and sprite facing direction
:param player\_x: player x coordinate
:param player\_y: player y coordinate
:param player\_angle:
:return:

\end{fulllineitems}


\end{fulllineitems}



\subsection{objects.GameObject module}
\label{\detokenize{objects:module-objects.GameObject}}\label{\detokenize{objects:objects-gameobject-module}}\index{module@\spxentry{module}!objects.GameObject@\spxentry{objects.GameObject}}\index{objects.GameObject@\spxentry{objects.GameObject}!module@\spxentry{module}}\index{GameObject (class in objects.GameObject)@\spxentry{GameObject}\spxextra{class in objects.GameObject}}

\begin{fulllineitems}
\phantomsection\label{\detokenize{objects:objects.GameObject.GameObject}}
\pysigstartsignatures
\pysigline{\sphinxbfcode{\sphinxupquote{class\DUrole{w,w}{  }}}\sphinxcode{\sphinxupquote{objects.GameObject.}}\sphinxbfcode{\sphinxupquote{GameObject}}}
\pysigstopsignatures
\sphinxAtStartPar
Bases: \sphinxcode{\sphinxupquote{ABC}}

\sphinxAtStartPar
abstract class for every object (entity) in a game.
Implemented according to Update Method design pattern
\index{render() (objects.GameObject.GameObject method)@\spxentry{render()}\spxextra{objects.GameObject.GameObject method}}

\begin{fulllineitems}
\phantomsection\label{\detokenize{objects:objects.GameObject.GameObject.render}}
\pysigstartsignatures
\pysiglinewithargsret{\sphinxbfcode{\sphinxupquote{abstract\DUrole{w,w}{  }}}\sphinxbfcode{\sphinxupquote{render}}}{\sphinxparam{\DUrole{n,n}{canvas}}}{}
\pysigstopsignatures
\end{fulllineitems}

\index{update() (objects.GameObject.GameObject method)@\spxentry{update()}\spxextra{objects.GameObject.GameObject method}}

\begin{fulllineitems}
\phantomsection\label{\detokenize{objects:objects.GameObject.GameObject.update}}
\pysigstartsignatures
\pysiglinewithargsret{\sphinxbfcode{\sphinxupquote{abstract\DUrole{w,w}{  }}}\sphinxbfcode{\sphinxupquote{update}}}{}{}
\pysigstopsignatures
\end{fulllineitems}


\end{fulllineitems}



\subsection{objects.Level module}
\label{\detokenize{objects:module-objects.Level}}\label{\detokenize{objects:objects-level-module}}\index{module@\spxentry{module}!objects.Level@\spxentry{objects.Level}}\index{objects.Level@\spxentry{objects.Level}!module@\spxentry{module}}\index{Level (class in objects.Level)@\spxentry{Level}\spxextra{class in objects.Level}}

\begin{fulllineitems}
\phantomsection\label{\detokenize{objects:objects.Level.Level}}
\pysigstartsignatures
\pysiglinewithargsret{\sphinxbfcode{\sphinxupquote{class\DUrole{w,w}{  }}}\sphinxcode{\sphinxupquote{objects.Level.}}\sphinxbfcode{\sphinxupquote{Level}}}{\sphinxparam{\DUrole{n,n}{level\_map}}, \sphinxparam{\DUrole{n,n}{screen\_height}}, \sphinxparam{\DUrole{n,n}{screen\_width}}}{}
\pysigstopsignatures
\sphinxAtStartPar
Bases: \sphinxcode{\sphinxupquote{object}}
\index{render() (objects.Level.Level method)@\spxentry{render()}\spxextra{objects.Level.Level method}}

\begin{fulllineitems}
\phantomsection\label{\detokenize{objects:objects.Level.Level.render}}
\pysigstartsignatures
\pysiglinewithargsret{\sphinxbfcode{\sphinxupquote{render}}}{\sphinxparam{\DUrole{n,n}{canvas}}}{}
\pysigstopsignatures
\sphinxAtStartPar
renders level in 2D view
:param canvas:
:return: updated canvas

\end{fulllineitems}

\index{update() (objects.Level.Level method)@\spxentry{update()}\spxextra{objects.Level.Level method}}

\begin{fulllineitems}
\phantomsection\label{\detokenize{objects:objects.Level.Level.update}}
\pysigstartsignatures
\pysiglinewithargsret{\sphinxbfcode{\sphinxupquote{update}}}{\sphinxparam{\DUrole{n,n}{player\_x}}, \sphinxparam{\DUrole{n,n}{player\_y}}, \sphinxparam{\DUrole{n,n}{player\_angle}}}{}
\pysigstopsignatures
\sphinxAtStartPar
updates all elements of the level that player can interact with. Right now only opens doors
:param player\_x:
:param player\_y:
:param player\_angle:
:return:

\end{fulllineitems}


\end{fulllineitems}



\subsection{objects.Player module}
\label{\detokenize{objects:module-objects.Player}}\label{\detokenize{objects:objects-player-module}}\index{module@\spxentry{module}!objects.Player@\spxentry{objects.Player}}\index{objects.Player@\spxentry{objects.Player}!module@\spxentry{module}}\index{Player (class in objects.Player)@\spxentry{Player}\spxextra{class in objects.Player}}

\begin{fulllineitems}
\phantomsection\label{\detokenize{objects:objects.Player.Player}}
\pysigstartsignatures
\pysiglinewithargsret{\sphinxbfcode{\sphinxupquote{class\DUrole{w,w}{  }}}\sphinxcode{\sphinxupquote{objects.Player.}}\sphinxbfcode{\sphinxupquote{Player}}}{\sphinxparam{\DUrole{n,n}{x}}, \sphinxparam{\DUrole{n,n}{y}}, \sphinxparam{\DUrole{n,n}{angle}}, \sphinxparam{\DUrole{n,n}{speed}}, \sphinxparam{\DUrole{n,n}{rotation\_speed}}, \sphinxparam{\DUrole{n,n}{fov}}, \sphinxparam{\DUrole{n,n}{vertical\_angle}}, \sphinxparam{\DUrole{n,n}{vision\_distance}}, \sphinxparam{\DUrole{n,n}{radius}}, \sphinxparam{\DUrole{n,n}{weapons}}}{}
\pysigstopsignatures
\sphinxAtStartPar
Bases: {\hyperref[\detokenize{objects:objects.Actor.Actor}]{\sphinxcrossref{\sphinxcode{\sphinxupquote{Actor}}}}}
\index{change\_weapon() (objects.Player.Player method)@\spxentry{change\_weapon()}\spxextra{objects.Player.Player method}}

\begin{fulllineitems}
\phantomsection\label{\detokenize{objects:objects.Player.Player.change_weapon}}
\pysigstartsignatures
\pysiglinewithargsret{\sphinxbfcode{\sphinxupquote{change\_weapon}}}{\sphinxparam{\DUrole{n,n}{index}}}{}
\pysigstopsignatures
\sphinxAtStartPar
changes weapon
:param index: weapon number
:return:

\end{fulllineitems}

\index{render() (objects.Player.Player method)@\spxentry{render()}\spxextra{objects.Player.Player method}}

\begin{fulllineitems}
\phantomsection\label{\detokenize{objects:objects.Player.Player.render}}
\pysigstartsignatures
\pysiglinewithargsret{\sphinxbfcode{\sphinxupquote{render}}}{\sphinxparam{\DUrole{n,n}{canvas}}, \sphinxparam{\DUrole{n,n}{map\_tile\_size}}}{}
\pysigstopsignatures
\sphinxAtStartPar
renders selected weapon
:param canvas:
:param map\_tile\_size:
:return: updated canvas

\end{fulllineitems}


\end{fulllineitems}



\subsection{objects.Sprite2D module}
\label{\detokenize{objects:module-objects.Sprite2D}}\label{\detokenize{objects:objects-sprite2d-module}}\index{module@\spxentry{module}!objects.Sprite2D@\spxentry{objects.Sprite2D}}\index{objects.Sprite2D@\spxentry{objects.Sprite2D}!module@\spxentry{module}}\index{Sprite2D (class in objects.Sprite2D)@\spxentry{Sprite2D}\spxextra{class in objects.Sprite2D}}

\begin{fulllineitems}
\phantomsection\label{\detokenize{objects:objects.Sprite2D.Sprite2D}}
\pysigstartsignatures
\pysiglinewithargsret{\sphinxbfcode{\sphinxupquote{class\DUrole{w,w}{  }}}\sphinxcode{\sphinxupquote{objects.Sprite2D.}}\sphinxbfcode{\sphinxupquote{Sprite2D}}}{\sphinxparam{\DUrole{n,n}{x}}, \sphinxparam{\DUrole{n,n}{y}}, \sphinxparam{\DUrole{n,n}{radius}}, \sphinxparam{\DUrole{n,n}{render\_radius}}, \sphinxparam{\DUrole{n,n}{path}}}{}
\pysigstopsignatures
\sphinxAtStartPar
Bases: {\hyperref[\detokenize{objects:objects.GameObject.GameObject}]{\sphinxcrossref{\sphinxcode{\sphinxupquote{GameObject}}}}}
\index{render() (objects.Sprite2D.Sprite2D method)@\spxentry{render()}\spxextra{objects.Sprite2D.Sprite2D method}}

\begin{fulllineitems}
\phantomsection\label{\detokenize{objects:objects.Sprite2D.Sprite2D.render}}
\pysigstartsignatures
\pysiglinewithargsret{\sphinxbfcode{\sphinxupquote{render}}}{\sphinxparam{\DUrole{n,n}{canvas}}, \sphinxparam{\DUrole{n,n}{x}}, \sphinxparam{\DUrole{n,n}{y}}, \sphinxparam{\DUrole{n,n}{width}}, \sphinxparam{\DUrole{n,n}{height}}, \sphinxparam{\DUrole{n,n}{brightness}}}{}
\pysigstopsignatures
\sphinxAtStartPar
rendering sprite
:param canvas:
:param x:
:param y:
:param width:
:param height:
:param brightness:
:return:

\end{fulllineitems}

\index{update() (objects.Sprite2D.Sprite2D method)@\spxentry{update()}\spxextra{objects.Sprite2D.Sprite2D method}}

\begin{fulllineitems}
\phantomsection\label{\detokenize{objects:objects.Sprite2D.Sprite2D.update}}
\pysigstartsignatures
\pysiglinewithargsret{\sphinxbfcode{\sphinxupquote{update}}}{\sphinxparam{\DUrole{n,n}{player\_x}}, \sphinxparam{\DUrole{n,n}{player\_y}}, \sphinxparam{\DUrole{n,n}{player\_angle}}}{}
\pysigstopsignatures
\end{fulllineitems}


\end{fulllineitems}



\subsection{objects.Texture module}
\label{\detokenize{objects:module-objects.Texture}}\label{\detokenize{objects:objects-texture-module}}\index{module@\spxentry{module}!objects.Texture@\spxentry{objects.Texture}}\index{objects.Texture@\spxentry{objects.Texture}!module@\spxentry{module}}\index{Texture (class in objects.Texture)@\spxentry{Texture}\spxextra{class in objects.Texture}}

\begin{fulllineitems}
\phantomsection\label{\detokenize{objects:objects.Texture.Texture}}
\pysigstartsignatures
\pysiglinewithargsret{\sphinxbfcode{\sphinxupquote{class\DUrole{w,w}{  }}}\sphinxcode{\sphinxupquote{objects.Texture.}}\sphinxbfcode{\sphinxupquote{Texture}}}{\sphinxparam{\DUrole{n,n}{texture\_path}}}{}
\pysigstopsignatures
\sphinxAtStartPar
Bases: \sphinxcode{\sphinxupquote{object}}
\index{create\_reversed() (objects.Texture.Texture method)@\spxentry{create\_reversed()}\spxextra{objects.Texture.Texture method}}

\begin{fulllineitems}
\phantomsection\label{\detokenize{objects:objects.Texture.Texture.create_reversed}}
\pysigstartsignatures
\pysiglinewithargsret{\sphinxbfcode{\sphinxupquote{create\_reversed}}}{}{}
\pysigstopsignatures
\sphinxAtStartPar
reverses the texture in x dimension

\end{fulllineitems}

\index{load() (objects.Texture.Texture method)@\spxentry{load()}\spxextra{objects.Texture.Texture method}}

\begin{fulllineitems}
\phantomsection\label{\detokenize{objects:objects.Texture.Texture.load}}
\pysigstartsignatures
\pysiglinewithargsret{\sphinxbfcode{\sphinxupquote{load}}}{}{}
\pysigstopsignatures
\sphinxAtStartPar
loading texture from file to the array of pixels in rgb format

\end{fulllineitems}

\index{render() (objects.Texture.Texture method)@\spxentry{render()}\spxextra{objects.Texture.Texture method}}

\begin{fulllineitems}
\phantomsection\label{\detokenize{objects:objects.Texture.Texture.render}}
\pysigstartsignatures
\pysiglinewithargsret{\sphinxbfcode{\sphinxupquote{render}}}{\sphinxparam{\DUrole{n,n}{canvas}}}{}
\pysigstopsignatures
\sphinxAtStartPar
rendering texture in 2D view (only for testing)
:param canvas:
:return: updated canvas

\end{fulllineitems}


\end{fulllineitems}



\subsection{objects.TextureStripe module}
\label{\detokenize{objects:module-objects.TextureStripe}}\label{\detokenize{objects:objects-texturestripe-module}}\index{module@\spxentry{module}!objects.TextureStripe@\spxentry{objects.TextureStripe}}\index{objects.TextureStripe@\spxentry{objects.TextureStripe}!module@\spxentry{module}}\index{TextureStripe (class in objects.TextureStripe)@\spxentry{TextureStripe}\spxextra{class in objects.TextureStripe}}

\begin{fulllineitems}
\phantomsection\label{\detokenize{objects:objects.TextureStripe.TextureStripe}}
\pysigstartsignatures
\pysiglinewithargsret{\sphinxbfcode{\sphinxupquote{class\DUrole{w,w}{  }}}\sphinxcode{\sphinxupquote{objects.TextureStripe.}}\sphinxbfcode{\sphinxupquote{TextureStripe}}}{\sphinxparam{\DUrole{n,n}{segments}}}{}
\pysigstopsignatures
\sphinxAtStartPar
Bases: \sphinxcode{\sphinxupquote{object}}
\index{render() (objects.TextureStripe.TextureStripe method)@\spxentry{render()}\spxextra{objects.TextureStripe.TextureStripe method}}

\begin{fulllineitems}
\phantomsection\label{\detokenize{objects:objects.TextureStripe.TextureStripe.render}}
\pysigstartsignatures
\pysiglinewithargsret{\sphinxbfcode{\sphinxupquote{render}}}{\sphinxparam{\DUrole{n,n}{canvas}}}{}
\pysigstopsignatures
\sphinxAtStartPar
creates rectangles on canvas representing pixels of texture
:param canvas:
:return: updated canvas

\end{fulllineitems}


\end{fulllineitems}



\subsection{objects.Weapon module}
\label{\detokenize{objects:module-objects.Weapon}}\label{\detokenize{objects:objects-weapon-module}}\index{module@\spxentry{module}!objects.Weapon@\spxentry{objects.Weapon}}\index{objects.Weapon@\spxentry{objects.Weapon}!module@\spxentry{module}}\index{Weapon (class in objects.Weapon)@\spxentry{Weapon}\spxextra{class in objects.Weapon}}

\begin{fulllineitems}
\phantomsection\label{\detokenize{objects:objects.Weapon.Weapon}}
\pysigstartsignatures
\pysiglinewithargsret{\sphinxbfcode{\sphinxupquote{class\DUrole{w,w}{  }}}\sphinxcode{\sphinxupquote{objects.Weapon.}}\sphinxbfcode{\sphinxupquote{Weapon}}}{\sphinxparam{\DUrole{n,n}{damage}}, \sphinxparam{\DUrole{n,n}{speed}}, \sphinxparam{\DUrole{n,n}{auto}}, \sphinxparam{\DUrole{n,n}{image\_path}}}{}
\pysigstopsignatures
\sphinxAtStartPar
Bases: \sphinxcode{\sphinxupquote{object}}
\index{render() (objects.Weapon.Weapon method)@\spxentry{render()}\spxextra{objects.Weapon.Weapon method}}

\begin{fulllineitems}
\phantomsection\label{\detokenize{objects:objects.Weapon.Weapon.render}}
\pysigstartsignatures
\pysiglinewithargsret{\sphinxbfcode{\sphinxupquote{render}}}{\sphinxparam{\DUrole{n,n}{canvas}}}{}
\pysigstopsignatures
\sphinxAtStartPar
rendering weapon on screen
:param canvas:
:return:

\end{fulllineitems}

\index{update() (objects.Weapon.Weapon method)@\spxentry{update()}\spxextra{objects.Weapon.Weapon method}}

\begin{fulllineitems}
\phantomsection\label{\detokenize{objects:objects.Weapon.Weapon.update}}
\pysigstartsignatures
\pysiglinewithargsret{\sphinxbfcode{\sphinxupquote{update}}}{}{}
\pysigstopsignatures
\end{fulllineitems}


\end{fulllineitems}



\subsection{Module contents}
\label{\detokenize{objects:module-objects}}\label{\detokenize{objects:module-contents}}\index{module@\spxentry{module}!objects@\spxentry{objects}}\index{objects@\spxentry{objects}!module@\spxentry{module}}
\sphinxstepscope


\section{settings module}
\label{\detokenize{settings:module-settings}}\label{\detokenize{settings:settings-module}}\label{\detokenize{settings::doc}}\index{module@\spxentry{module}!settings@\spxentry{settings}}\index{settings@\spxentry{settings}!module@\spxentry{module}}
\sphinxAtStartPar
setting all necessary parameters and key bindings

\sphinxstepscope


\section{utils module}
\label{\detokenize{utils:module-utils}}\label{\detokenize{utils:utils-module}}\label{\detokenize{utils::doc}}\index{module@\spxentry{module}!utils@\spxentry{utils}}\index{utils@\spxentry{utils}!module@\spxentry{module}}\index{hex\_to\_rgb() (in module utils)@\spxentry{hex\_to\_rgb()}\spxextra{in module utils}}

\begin{fulllineitems}
\phantomsection\label{\detokenize{utils:utils.hex_to_rgb}}
\pysigstartsignatures
\pysiglinewithargsret{\sphinxcode{\sphinxupquote{utils.}}\sphinxbfcode{\sphinxupquote{hex\_to\_rgb}}}{\sphinxparam{\DUrole{n,n}{value}}}{}
\pysigstopsignatures
\sphinxAtStartPar
returns tuple of rgb values
:param value: hex string
:return: rgb tuple

\end{fulllineitems}

\index{return\_rotated\_actor\_position() (in module utils)@\spxentry{return\_rotated\_actor\_position()}\spxextra{in module utils}}

\begin{fulllineitems}
\phantomsection\label{\detokenize{utils:utils.return_rotated_actor_position}}
\pysigstartsignatures
\pysiglinewithargsret{\sphinxcode{\sphinxupquote{utils.}}\sphinxbfcode{\sphinxupquote{return\_rotated\_actor\_position}}}{\sphinxparam{\DUrole{n,n}{actor\_x}}, \sphinxparam{\DUrole{n,n}{actor\_y}}, \sphinxparam{\DUrole{n,n}{angle\_to\_rotate}}, \sphinxparam{\DUrole{n,n}{max\_x}}, \sphinxparam{\DUrole{n,n}{max\_y}}}{}
\pysigstopsignatures
\sphinxAtStartPar
rotates actor around the center of the map
:param actor\_x:
:param actor\_y:
:param angle\_to\_rotate:
:param max\_x:
:param max\_y:
:return: x\_rotated, y\_rotated

\end{fulllineitems}

\index{return\_rotated\_matrix() (in module utils)@\spxentry{return\_rotated\_matrix()}\spxextra{in module utils}}

\begin{fulllineitems}
\phantomsection\label{\detokenize{utils:utils.return_rotated_matrix}}
\pysigstartsignatures
\pysiglinewithargsret{\sphinxcode{\sphinxupquote{utils.}}\sphinxbfcode{\sphinxupquote{return\_rotated\_matrix}}}{\sphinxparam{\DUrole{n,n}{matrix}}}{}
\pysigstopsignatures
\sphinxAtStartPar
funtion that returns rotated matrix by 90 degrees
:param matrix:
:return:

\end{fulllineitems}

\index{rgb\_to\_hex() (in module utils)@\spxentry{rgb\_to\_hex()}\spxextra{in module utils}}

\begin{fulllineitems}
\phantomsection\label{\detokenize{utils:utils.rgb_to_hex}}
\pysigstartsignatures
\pysiglinewithargsret{\sphinxcode{\sphinxupquote{utils.}}\sphinxbfcode{\sphinxupquote{rgb\_to\_hex}}}{\sphinxparam{\DUrole{n,n}{rgb}}}{}
\pysigstopsignatures
\sphinxAtStartPar
returns color value string in hexadecimal format
:param rgb: rgb tuple
:return:

\end{fulllineitems}



\chapter{Indices and tables}
\label{\detokenize{index:indices-and-tables}}\begin{itemize}
\item {} 
\sphinxAtStartPar
\DUrole{xref,std,std-ref}{genindex}

\item {} 
\sphinxAtStartPar
\DUrole{xref,std,std-ref}{modindex}

\item {} 
\sphinxAtStartPar
\DUrole{xref,std,std-ref}{search}

\end{itemize}


\renewcommand{\indexname}{Python Module Index}
\begin{sphinxtheindex}
\let\bigletter\sphinxstyleindexlettergroup
\bigletter{c}
\item\relax\sphinxstyleindexentry{commands}\sphinxstyleindexpageref{commands:\detokenize{module-commands}}
\item\relax\sphinxstyleindexentry{commands.ChangeWeaponCommand}\sphinxstyleindexpageref{commands:\detokenize{module-commands.ChangeWeaponCommand}}
\item\relax\sphinxstyleindexentry{commands.Command}\sphinxstyleindexpageref{commands:\detokenize{module-commands.Command}}
\item\relax\sphinxstyleindexentry{commands.MoveActorCommand}\sphinxstyleindexpageref{commands:\detokenize{module-commands.MoveActorCommand}}
\item\relax\sphinxstyleindexentry{commands.RotateActorCommand}\sphinxstyleindexpageref{commands:\detokenize{module-commands.RotateActorCommand}}
\item\relax\sphinxstyleindexentry{commands.ZoomCommand}\sphinxstyleindexpageref{commands:\detokenize{module-commands.ZoomCommand}}
\indexspace
\bigletter{g}
\item\relax\sphinxstyleindexentry{GameEngine}\sphinxstyleindexpageref{GameEngine:\detokenize{module-GameEngine}}
\indexspace
\bigletter{i}
\item\relax\sphinxstyleindexentry{InputHandler}\sphinxstyleindexpageref{InputHandler:\detokenize{module-InputHandler}}
\indexspace
\bigletter{o}
\item\relax\sphinxstyleindexentry{objects}\sphinxstyleindexpageref{objects:\detokenize{module-objects}}
\item\relax\sphinxstyleindexentry{objects.Actor}\sphinxstyleindexpageref{objects:\detokenize{module-objects.Actor}}
\item\relax\sphinxstyleindexentry{objects.DirectionalSprite}\sphinxstyleindexpageref{objects:\detokenize{module-objects.DirectionalSprite}}
\item\relax\sphinxstyleindexentry{objects.GameObject}\sphinxstyleindexpageref{objects:\detokenize{module-objects.GameObject}}
\item\relax\sphinxstyleindexentry{objects.Level}\sphinxstyleindexpageref{objects:\detokenize{module-objects.Level}}
\item\relax\sphinxstyleindexentry{objects.Player}\sphinxstyleindexpageref{objects:\detokenize{module-objects.Player}}
\item\relax\sphinxstyleindexentry{objects.Sprite2D}\sphinxstyleindexpageref{objects:\detokenize{module-objects.Sprite2D}}
\item\relax\sphinxstyleindexentry{objects.Texture}\sphinxstyleindexpageref{objects:\detokenize{module-objects.Texture}}
\item\relax\sphinxstyleindexentry{objects.TextureStripe}\sphinxstyleindexpageref{objects:\detokenize{module-objects.TextureStripe}}
\item\relax\sphinxstyleindexentry{objects.Weapon}\sphinxstyleindexpageref{objects:\detokenize{module-objects.Weapon}}
\indexspace
\bigletter{r}
\item\relax\sphinxstyleindexentry{RaycastingEngine}\sphinxstyleindexpageref{RaycastingEngine:\detokenize{module-RaycastingEngine}}
\indexspace
\bigletter{s}
\item\relax\sphinxstyleindexentry{settings}\sphinxstyleindexpageref{settings:\detokenize{module-settings}}
\item\relax\sphinxstyleindexentry{SpriteRender}\sphinxstyleindexpageref{SpriteRender:\detokenize{module-SpriteRender}}
\indexspace
\bigletter{u}
\item\relax\sphinxstyleindexentry{utils}\sphinxstyleindexpageref{utils:\detokenize{module-utils}}
\end{sphinxtheindex}

\renewcommand{\indexname}{Index}
\printindex
\end{document}